\chapter{Introdução}
	\section{Considerações Iniciais e Objetivos}
		\subsection{Objetivos}
			\par Comparar o desempenho das estratégias de \textit{feature learnning} baseadas em \textit{autoencoders} com técnicas de análise como as \textit{Wavelet-Packet} de Tempo Discreto usando a engenharia paraconsistente de características.
			\par O problema-alvo deste projeto é conceber algoritmos biométricos para autenticar, em princípio por meio da fala,  indivíduos com locuções severamente degradadas complementando tais informações com aquelas provenientes dos sinais cerebrais extraídos durante a fonação (imagined speech).
			\par Estudar base de dados existentes, criar a própria base de dados. Depois iniciar com a criação de vetores de características usando \textit{autoencoders} e análise com \textit{wavelets packet tranform}, e classificá-los com redes neurais profundas residuais e recorrentes.
			\par Text-dependent: Uma mesma frase é falada e imaginada
			\par Text-independent: Pode ser falada qualquer coisa tanto no treinamento quanto nos testes.
	\section{Estrutura do trabalho}
