\chapter{Introdução}
	\section{Considerações Iniciais e Objetivos}
	
		\par A autenticação baseada em biometria vem despertando um interesse crescente, uma vez que se aproveita de características biométricas altamente correlacionadas com o indivíduo, resultando em um nível mais elevado de segurança. Algumas biometrias mostraram-se discriminatórias em relação a certos grupos, como os indivíduos com deficiências físicas, que podem enfrentar dificuldades ao utilizá-las. Esse é o caso de boa parte dos tipos de biometrias utilizadas, incluindo a fala que é um dos focos desse estudo. Por outro lado, o eletroencefalograma (EEG) evita tais problemas, ao mesmo tempo em que acrescenta algumas características únicas, como a exploração das características neurais do usuário, criando, assim, uma contramedida contra pressões externas e complementando eventuais falhas ou faltas de outros métodos.
	
		\subsection{Objetivos}
			\par Comparar o desempenho das estratégias de \textit{feature learnning} baseadas em \textit{autoencoders} com técnicas de análise como as \textit{Wavelet-Packet} de Tempo Discreto usando a engenharia paraconsistente de características.
			\par O problema-alvo deste projeto é conceber algoritmos biométricos para autenticar, em princípio por meio da fala,  indivíduos com locuções severamente degradadas complementando tais informações com aquelas provenientes dos sinais cerebrais extraídos durante a fonação (imagined speech).
			\par Estudar base de dados existentes, criar a própria base de dados. Depois iniciar com a criação de vetores de características usando \textit{autoencoders} e análise com \textit{wavelets packet tranform}, e classificá-los com redes neurais profundas residuais e recorrentes.
			\par Text-dependent: Uma mesma frase é falada e imaginada
			\par Text-independent: Pode ser falada qualquer coisa tanto no treinamento quanto nos testes.
	\section{Estrutura do trabalho}
			\par No \autoref{ch:revisao} a definição dos conceitos utilizados é feita na \autoref{sec:conceitos}, uma revisão do estado-da-arte se encontra em \autoref{sec:trabalhosMaisRecentes}. O \autoref{chap:propApproach} define a abordagem proposta e é dividido em: Protocolo de coleta e organização da base de sinais (\autoref{sec:baseDeSinais}), Estrutura da estratégia proposta (\autoref{sec:estruturaDaEstrategiaProposta}) e por fim o cronograma previsto (\autoref{sec:cronograma}).