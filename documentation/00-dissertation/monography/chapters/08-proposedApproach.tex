\chapter{Abordagem proposta} \label{chap:propApproach}
	\section{A Base de sinais}
		\par Serão coletadas, de forma isolada, as falas imaginadas, as falas fonadas e as falas simultaneamente imaginadas e fonadas, resultando em três bases de dados distintas permitindo comparar os dois tipos de sinais e suas interações com os sensores, tanto quando isolados quanto combinados. Assim, será possível obter dados sobre a fala imaginada sem a interferência dos músculos relacionados à fala, além de dados sobre a fala efetivamente pronunciada e a fala imaginada.
		
		\par O objetivo é que haja uma ampla gama de dados para possibilitar pesquisas futuras tanto no campo da fala imaginada quanto na fala fonada, ou em ambas as áreas.
		
	    \subsection{Protocolo de Coleta de Dados}
		    
		    \par Os protocolos de coleta especificam dois ciclos: Um chamado de \textbf{ruidoso} esquematizado na \autoref{fig:coletaRuidosa}, no qual, durante a captura, serão reproduzidos ruídos comuns de lugares com pessoas conversando gravados previamente. E outro chamado \textbf{silencioso}, como mostrado na \autoref{fig:coletaSilenciosa}, no qual a captura se dará, na medida da disponibilidade dos equipamentos disponíveis, com o menor ruído ambiente possível. Dessa forma também se espera conseguir avaliar qual o impacto dessa variante na produção e classificação dos dados.
		    
		    \subsection{Sentenças coletadas}
		    
			    \par As sentenças coletadas, tanto fonadas quanto imaginadas, serão as seguintes:
			    \begin{itemize}
			    	\item \textbf{palavras:} Primeiro nome da/do voluntária(o), cima, baixo, esquerda, direita, senha inventada
			    	\item \textbf{frases:} Estou com fome, Sinto dor, Estou com sede, Estou com sono, Entrar no sistema.
			    \end{itemize}
			    
			    \par Tais sentenças foram escolhidas pois representam comandos básicos úteis em cenários que vão desde a autenticação da pessoa até comandos básicos que podem ser úteis em cenários diversos, muitos dos quais foram aproveitados e/ou inspirados pelos trabalhos revisados na \autoref{sec:trabalhosMaisRecentes}.
		    
		    \subsubsection{Questionário preliminar}
		    
			    \par Antes da coleta dos sinais de voz e neurológicos será feito um questionário com as seguintes perguntas:
			    
			    \begin{itemize}
			    	\item ingeriu ou ingere algum medicamento? Há quanto tempo?
			    	\item usou alguma droga legal ou ilegal? Há quanto tempo?
			    	\item ingeriu alguma bebida energética? Há quanto tempo?
			    	\item qual sua etnia?
			    	\item qual seu sexo?
			    	\item qual seu gênero?
			    	\item é canhota, destra ou ambidestra?
			    \end{itemize}
			    
			\subsubsection{Taxa de amostragem}
			
				\par Embora, em ratos, neurônios biológicos saudáveis possam proporcionar leituras de EEG de até 2Khz, tal fenômeno não é documentado em cérebros humanos saudáveis e estudos anteriores indicam que leituras acima de 120Hz ocorrem em cérebros de pessoas epiléticas \cite{Moffett2017}. No entanto, apenas a ocorrência de altas frequências, não indica necessariamente uma patologia já que na região mais externa do telencéfalo conhecida como neocórtex podem ocorrer oscilações rápidas que podem atingir em torno de 200Hz \cite{hfreOscEngel}.
				
				\par Portanto, segundo o teorema de Nyquist, a taxa de amostragem mínima é de 400Hz, porém, dadas as possibilidades aventadas, há que se adotar uma certa margem de segurança, sendo assim, escolheu-se uma taxa de amostragem inicial de 800Hz.
				
				%TODO subamostrar depois?
				
		    \subsubsection{Protocolo de coleta da fala fonada}
		    	\label{sec:coletaFalaFonada}
			    \begin{enumerate}
			    	\item \textbf{ciclo silencioso.}
			    	\begin{enumerate}
			    		\item em uma tela é exibido ao participante por 5 segundos a sentença que deve ser pronunciada.\label{itm:exibePalavraFalaSilencio}
			    		\item um sinal é tocado por 1 segundo
			    		\item por 5 segundos a sentença deve ser \textbf{pronunciada} uma única vez.
			    		\item volta para o \autoref{itm:exibePalavraFalaSilencio} até que todas as sentenças sejam pronunciadas.
			    	\end{enumerate}
			    	
			    	\item \textbf{ciclo ruidoso}.
			    	\begin{enumerate}
			    		\item inicia-se a reprodução do ruído
			    		\item em uma tela é exibido ao participante por 5 segundos a sentença que deve ser pronunciada.\label{itm:exibePalavraFalaRuido}
			    		\item um sinal é tocado por 1 segundo
			    		\item por 5 segundos a sentença deve ser \textbf{pronunciada} uma única vez.
			    		\item volta para o \autoref{itm:exibePalavraFalaRuido} até que todas as sentenças sejam pronunciadas.
			    		\item a reprodução do ruído é finalizada.
			    	\end{enumerate}
			    \end{enumerate}
		    
		    \subsubsection{Protocolo de coleta da fala imaginada}
		    	\label{sec:coletaFalaImaginada}
  			    \par No caso específico da fala imaginada serão coletados 5 segundos de sinais cerebrais dos voluntários antes que qualquer um dos protocolos citados sejam executados, se espera dessa forma captar um padrão que talvez se modifique devido a tensão emocional causada pelos sinais sonoros e ordens dadas.
  			    
  			    \par Afim de evitar as interferências intrínsecas à captação do sinal o mesmo deve passar por um \textbf{filtro passa banda de 1 a 800Hz} \cite{JALALYBIDGOLY2020101788}, \cite{Moffett2017}, \cite{hfreOscEngel}, e outro para excluir a frequência de 60Hz, típica da rede elétrica brasileira, mantendo assim, por uma boa margem, as frequências operacionais típicas de um cérebro humano .
  			    %TODO \par Uma normalização dos valores também se faz necessária, 
		    	
			    \begin{enumerate}
			    	\item \textbf{ciclo silencioso.}
			    	\begin{enumerate}
			    		\item em uma tela é exibido ao participante por 5 segundos a sentença que deve ser imaginada.\label{itm:exibePalavraImaginaSilencio}
			    		\item um sinal é tocado por 1 segundo
			    		\item Por 5 segundos a sentença deve ser \textbf{imaginada} uma única vez.
			    		\item volta para o \autoref{itm:exibePalavraImaginaSilencio} até que todas as sentenças sejam imaginadas.
			    	\end{enumerate}
			    	\item \textbf{ciclo ruidoso}.
			    	\begin{enumerate}
			    		\item inicia-se a reprodução do ruído
			    		\item em uma tela é exibido ao participante por 5 segundos a sentença que deve ser imaginada.\label{itm:exibePalavraImaginaRuido}
			    		\item um sinal é tocado por 1 segundo
			    		\item por 5 segundos a sentença deve ser \textbf{imaginada} uma única vez.
			    		\item volta para o \autoref{itm:exibePalavraImaginaRuido} até que todas as sentenças sejam pronunciadas.
			    		\item a reprodução do ruído é finalizada.
			    	\end{enumerate}
			    \end{enumerate}
			    
				\begin{figure}[H]
					\centering
					\caption[Protocolo de coleta de dados]{Protocolo de coleta de dados aplicado as falas fonadas, imaginadas e mistas}
					\subfloat[Protocolo de coleta silenciosa]{
						\begin{tikzpicture}[transform shape]
	% Definindo estilos de nós
	
	% Início
	\tikzstyle{startstop} = [circle, rounded corners, minimum width=1cm, minimum height=1cm, text centered, fill=orange, text=black, font=\sffamily\bfseries]
	
	% Processo
	\tikzstyle{process} = [rectangle, rounded corners, minimum width=3cm, minimum height=1cm, text centered, fill=green, font=\sffamily\bfseries, inner sep=0.4cm]
	
	% Decisão
	\tikzstyle{decision} = [diamond, minimum width=3cm, minimum height=1cm, text centered, fill=yellow, text=black, font=\sffamily\bfseries, aspect=3]
	
	% Estilos das setas
	\tikzstyle{arrow} = [thick,->,>=stealth, draw=gray, line width=1.5mm]
	
	% Nós
	\node (start) [startstop] {Início};
	\node (display) [process, below=1cm of start] {Exibe próxima sentença por 5s};
	\node (signal) [process, right=1cm of display] {Sinal por 1s};
	\node (pronounce) [process, below=1cm of signal] {Captação por 5s};
	\node (loop) [decision, left=1cm of pronounce, below=1.5cm of display] {Existem mais sentenças?};
	\node (stop) [startstop, below=1.5cm of loop] {Fim};
	
	
	% Setas com curvas
	\draw [arrow] (start) to (display);
	\draw [arrow] (display) to (signal);
	\draw [arrow] (signal) to (pronounce);
	\draw [arrow, bend left=20] (pronounce) to (loop.east);
	\draw [arrow, bend left=20] (loop.north) to node[midway, sloped, below, font=\sffamily\bfseries, text=white] {Sim} (display);
	\draw [arrow, bend left=20] (loop.south) to node[midway, sloped, below, font=\sffamily\bfseries, text=white] {Não} (stop);
	
	% Agrupando o diagrama em um escopo
	\begin{scope}[on background layer]
		% Retângulo ao redor do diagrama com fundo azul
		\node[rectangle, fill=none, fit={(start) (display) (pronounce) (loop) (stop)}, inner sep=0.5cm] (background) {};
	\end{scope}
\end{tikzpicture}
						\label{fig:coletaSilenciosa}
					}
					\subfloat[Protocolo de coleta ruidosa]{%
						\begin{tikzpicture}[transform shape, scale=0.6]
	% Definindo estilos de nós
	
	% Início
	\tikzstyle{startstop} = [circle, rounded corners, minimum width=1cm, minimum height=1cm, text centered, fill=orange, text=black, font=\sffamily\bfseries]
	
	% Processo
	\tikzstyle{process} = [rectangle, rounded corners, minimum width=3cm, minimum height=1cm, text centered, fill=green, font=\sffamily\bfseries, inner sep=0.4cm]
	
	% Decisão
	\tikzstyle{decision} = [diamond, minimum width=3cm, minimum height=1cm, text centered, fill=yellow, text=black, font=\sffamily\bfseries, aspect=3]
	
	% Estilos das setas
	\tikzstyle{arrow} = [thick,->,>=stealth, draw=white, line width=1.5mm]
	
	% Nós
	\node (start) [startstop] {Início};
	
	\node (noise) [process, left=1cm of start] {Reproduzir ruído};
	
	\node (display) [process, below=1cm of noise] {Exibe próxima sentença por 5s};
	\node (signal) [process, right=1cm of display] {Sinal por 1s};
	\node (pronounce) [process, below=1cm of signal] {Captação por 5s};
	\node (loop) [decision, left=1cm of pronounce, below=1.8cm of display] {Existem mais sentenças?};
	
	\node (stopNoise) [process, below=1.4cm of loop] {Silenciar ruído};
	
	\node (stop) [startstop, right=1.5cm of stopNoise] {Fim};
	
	
	% Setas com curvas
	\draw [arrow] (start) to (noise);
	\draw [arrow] (noise) to (display);
	\draw [arrow] (display) to (signal);
	\draw [arrow] (signal) to (pronounce);
	\draw [arrow, bend left=20] (pronounce) to (loop.east);
	\draw [arrow, bend left=20] (loop.north) to node[midway, sloped, below, font=\sffamily\bfseries, text=white] {Sim} (display);
	
	\draw [arrow, bend left=20] (loop.south) to node[midway, sloped, below, font=\sffamily\bfseries, text=white] {Não} (stopNoise);
	
	\draw [arrow] (stopNoise) to (stop);
	
	% Agrupando o diagrama em um escopo
	\begin{scope}[on background layer]
		% Retângulo ao redor do diagrama com fundo azul
		\node[rectangle, fill=blue, fit={(start) (display) (pronounce) (loop) (stop)}, inner sep=0.5cm] (background) {};
	\end{scope}
\end{tikzpicture}
						\label{fig:coletaRuidosa}
					}
					\legend{Fonte: Elaborado pelo autor}
				\end{figure}
				
			\subsubsection{Protocolo de coleta da fala mista}
				\par Deve seguir os mesmos protocolos descritos anteriormente (\autoref{sec:coletaFalaFonada}, \autoref{sec:coletaFalaImaginada}) porém captando ambos os sinais fonados e imaginados simultaneamente;
				

	    \subsection{Organização da base de sinais}
	    	\par A organização da base dados deve se dividir em três categorias:
	    	\begin{itemize}
	    		\item fonadas
	    		\begin{itemize}
	    			\item com ruído
	    			\item sem ruído
	    		\end{itemize}
	    		\item imaginadas
	    		\begin{itemize}
	    			\item com ruído
	    			\item sem ruído
	    		\end{itemize}
	    		\item mista (fonadas e imaginadas simultaneamente)
			    \begin{itemize}
	    			\item com ruído
	    			\item sem ruído
	    		\end{itemize}
	    	\end{itemize}

			\par Cada registro deve ser vinculado a um código que deve referenciar os dados da pessoa que o gerou, a data, hora, minuto e segundo em que a fala iniciou e finalizou, a sentença e o tipo de fala (fonada ou imaginada).

	\section{Estrutura da estratégia proposta}
			
	
	\section{Procedimentos}

		
		\subsection{Tratamento do sinal}
			

	\subsection{Procedimento 01}