\chapter{Abordagem proposta} \label{chap:propApproach}
	\section{A Base de sinais}
	    \subsection{Coleta dos sinais}
	    \subsection{Organização da base de sinais}


	\section{Estrutura da estratégia proposta}
	
	\section{Procedimentos}
		\subsection{Protocolo de Coleta de Dados}
			\par O procedimento tem como objetivo a coleta de uma ampla gama de dados, começando com a gravação exclusiva das falas imaginadas. Posteriormente, serão registradas as falas pronunciadas. Esse método permitirá a obtenção de dados sobre a fala imaginada sem a interferência dos músculos relacionados à fala, além dos dados sobre a fala efetivamente pronunciada.
			
			\begin{enumerate}
				\item Primeiramente será exibido ao participante um sinal visual de espera por 5 segundos.
				
				\item Executar o ciclo silencioso. \label{itm:SilentCicle}
				\begin{enumerate}
					\item Por 5 segundos é exibida a palavra que deve ser pensada. 
					\item Por 5 segundos a palavra deve ser \textbf{imaginada} uma única vez.
					\item Um sinal visual é emitido novamente por 5 segundos
					\item Por 5 segundos a palavra deve ser \textbf{pronunciada} uma única vez.
				\end{enumerate}
				
				\item Executar o ciclo ruidoso.
				\begin{enumerate}
					\item Por 5 segundos é exibida a \textbf{mesma} palavra que deve ser pensada e sons de ambientes ruidosos começam a ser reproduzidos.
					\item Por 5 segundos a palavra deve ser \textbf{imaginada} uma única vez.
					\item Um sinal visual é emitido novamente por 5 segundos
					\item Por 5 segundos a palavra deve ser \textbf{pronunciada} uma única vez.
					\item A reprodução do ruído é finalizada.
				\end{enumerate}
				
				\item Se deve selecionar uma nova palavra e voltar para o \autoref{itm:SilentCicle}.
			\end{enumerate}
		
		\subsection{Tratamento do sinal}
			\par Afim de evitar as interferências intrínsecas à captação do sinal o mesmo deve passar por um \textbf{filtro passa baixa de 100Hz} mantendo por uma boa margem as frequências operacionais típicas de um cérebro humano \cite{JALALYBIDGOLY2020101788}.
			%TODO \par Uma normalização dos valores também se faz necessária, 

	\subsection{Procedimento 01}