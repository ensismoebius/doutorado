\begin{frame}
	\frametitle{Interfaces Humano-Máquina e EEG}
	
	\only<1>{
		\framesubtitle{EEG e as frequências do cérebro}
		\begin{columns}
			\column{.4\linewidth}
			\par \textbf{Eletroencefalograma}:
			\begin{itemize}
				\item não invasivo
				\item mais sujeito a ruídos
				\item exige tolerância a ruído
				\item mais econômico e simples de implementar
				\item eletrodos secos (reutilizáveis, mais interferência)
				\item eletrodos úmidos (não reutilizáveis, menos interferência)
			\end{itemize}
			\column{.6\linewidth}
			\begin{itemize}
				\item \textbf{Delta (1–4Hz)}: A onda mais lenta. Observada em bebês e durante o sono profundo em adultos.
				
				\item \textbf{Theta (4–8Hz)}: Observada em crianças, adultos sonolentos e durante a recordação de memórias.
				
				\item \textbf{Alpha (8–12Hz)}: Geralmente a banda de frequência dominante, aparecendo durante a consciência relaxada ou quando os olhos estão fechados.
				
				\item \textbf{Beta (12–25Hz)}: Associada ao pensamento, concentração ativa e atenção focada.
				
				\item \textbf{Gamma (acima de 25Hz)}: Observada durante o processamento sensorial múltiplo.
			\end{itemize}
		\end{columns}
	}
	
	\only<2>{
		\framesubtitle{Sistema 10-20}
		\begin{columns}
			\column{.4\linewidth}
				\par Posicionamento dos eletrodos de acordo com o padrão 10-20. Números ímpares são atribuídos aos eletrodos no hemisfério esquerdo, e números pares são atribuídos aos eletrodos no hemisfério direito \cite{sistema10-20}, \cite{JALALYBIDGOLY2020101788}.
			\column{.6\linewidth}
				\begin{figure}[h]
					\centering
					\includegraphics[width=1\linewidth]{../monography/images/10–20StandardAndLobes}
					\label{fig:1020standardandlobes}
				\end{figure}
		\end{columns}
	}
\end{frame}