\begin{frame}
	\frametitle{Filtros digitais \textit{wavelet}}
	\only<1>{
		\framesubtitle{Propriedades}
		\begin{itemize}
			\item suporte compacto.
			\item análise multirresolução.
			\item wavelet regular e wavelet packet.
			\item variadas funções-filtro não-periódicas.
			\item análise detalhada em altas e baixas frequências.
		\end{itemize}
	}
	\only<2>{
		\framesubtitle{Restrição de escopo}
		\begin{itemize}
			\item wavelet packet
			\item domínio discreto.
			\item apenas transformadas diretas.
			\item não haverá reconstrução do sinal.
			\item construção dos vetores de características.
		\end{itemize}
	}

	\only<3>{
		\framesubtitle{Resposta em frequência e linearidade}
		\input{../monography/tables/someWaveletsProperties.tex}
	}

	\only<4>{
		\framesubtitle{Algoritmo de Malat}
		\begin{itemize}
			\item \textit{Wavelet} Haar: $h[\cdot] = [\frac{1}{\sqrt{2}}, \frac{1}{\sqrt{2}}]$.
			\item Par ortogonal: $g[\cdot] = [\frac{1}{\sqrt{2}}, \frac{-1}{\sqrt{2}}]$.
			\item sinal: $s[\cdot] = [1,2,3,4]$.
		\end{itemize}
		
		\begin{equation*}
			\begin{pmatrix}
			\frac{1}{\sqrt{2}}, \frac{1}{\sqrt{2}}, 0, 0\\
			\frac{1}{\sqrt{2}}, \frac{-1}{\sqrt{2}}, 0, 0\\
			0, 0, \frac{1}{\sqrt{2}}, \frac{1}{\sqrt{2}}\\
			0, 0, \frac{1}{\sqrt{2}}, \frac{1}{\sqrt{2}}\\
			\end{pmatrix} 
			\cdot
			\begin{pmatrix}
			1\\
			2\\
			3\\
			4\\
			\end{pmatrix} 
			=
			\begin{pmatrix}
			\frac{3}{\sqrt{2}}\\
			\frac{-1}{\sqrt{2}}\\
			\frac{7}{\sqrt{2}}\\
			\frac{-1}{\sqrt{2}}\\
			\end{pmatrix}
			\Rightarrow \Big[
			\frac{3}{\sqrt{2}},
			\frac{7}{\sqrt{2}},
			\frac{-1}{\sqrt{2}},
			\frac{-1}{\sqrt{2}}
			\Big]\qquad.
		\end{equation*}
	}

	\only<5>{
		\framesubtitle{Exemplo de \textit{wavelet} regular e \textit{packet}}
		\begin{figure}
			\centering
			\includegraphics[width=\linewidth]{../monography/images/haarWaveletExamples}
		\end{figure}
	}

	\only<6>{
		\framesubtitle{Porquê e \textit{wavelet packet}?}
		\begin{itemize}
			\item decompõe a aproximação e os detalhes
			\item proporciona um nível de decomposição maior
		\end{itemize}
	}
\end{frame}







