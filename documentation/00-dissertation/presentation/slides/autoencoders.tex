\begin{frame}
	\frametitle{Autoencoders}
	\only<1>{
		\framesubtitle{Características de um autoencoder subcompleto}
		\begin{columns}
			\column{.3\linewidth}
				\par Exemplo esquemático de um \textit{autoencoder}: Os nós de entrada estão em cinza, a camada de código em amarelo contêm as características codificadas e finalmente a camada de reconstrução em vermelho contêm um cópia aproximada da entrada.
			\column{.7\linewidth}
				\begin{figure}[h]
					\centering
					\scalebox{.8}{
						\begin{tikzpicture}[scale=2.5]
	% Define scaling factors
	\def\inputscale{.5}
	\def\hiddenscale{.5}
	\def\outputscale{.5}
	
	%input layer
	\foreach \i in {0,1,2,3,4} {
		\node (input_node\i) at (0,\i*\inputscale) {}; 
		\filldraw[fill=gray] (input_node\i) circle (0.15cm);
		\node at (input_node\i) {\tiny $x_\i$};
		
		\draw[->,in=180,out=0] (-0.5,\i*\inputscale) to (input_node\i);
	}
	
	%hidden layer
	\foreach \i in {0,1,2} {
		\node (hidden_node\i) at (1.5,\i*\hiddenscale+.5) {}; 
		\filldraw[fill=yellow] (hidden_node\i) circle (0.15cm);
		\node at (hidden_node\i) {\tiny $h_\i$};
	}
	
	%output layer
	\foreach \i in {0,1,2,3,4} {
		\node (output_node\i) at (3,\i*\outputscale) {}; 
		\filldraw[fill=red] (output_node\i) circle (0.15cm);
		\node at (output_node\i) {\tiny $r_\i$};
	}
	
	% Connect layers
	\foreach \i in {0,1,2,3,4} {
		\foreach \j in {0,1,2} {
			\draw[->,in=180,out=0] (input_node\i) to (hidden_node\j);
			\draw[->,in=180,out=0] (hidden_node\j) to (output_node\i);
		}
	}
	
	%descriptions
	\draw[snake=brace,mirror snake,raise snake=45pt,brown] (-0.2,0.4) -- (0.3,0.4) node[black,midway,yshift=-50pt,below]{\tiny entrada};
	
	\draw[snake=brace,mirror snake,raise snake=45pt,brown] (1.3,0.8) -- (1.7,0.8) node[black,midway,yshift=-50pt,below]{\tiny código};
	
	\draw[snake=brace,mirror snake,raise snake=45pt,brown] (2.7,0.4) -- (3.3,0.4) node[black,midway,yshift=-50pt,below]{\tiny reconstrução};

\end{tikzpicture}
					}
					\label{fig:autoencoder2}
				\end{figure}
		\end{columns}
	}
	\only<2>{
		\framesubtitle{Denoising}
		\par Representação funcional de um \textit{denoising autoencoder}: Sendo o vetor $x$ uma entrada e $n$ um função que adiciona um ruído aleatório então a informação com ruído $\hat{x}$ é definida como $\hat{x} = n(x)$, portanto o vetor $h$ é resultado da aplicação de uma função codificadora $f$ sobre $x$: $h = f(x)$, finalmente $r$ é a reconstrução de $x$ a partir $h$ através de uma função decodificadora $g$: $r = g(h)$. É importante notar que $r$ é comparada com $x$ e não com $\hat{x}$.
		
		\begin{figure}
			\centering
			\begin{tikzpicture}[node distance=3cm, every edge/.style={draw=black,->}]
	% Nodes
	\node[draw, rectangle, fill=blue] (x) {$x$};
	
	\node[draw, rectangle, above right=-.6cm and 1cm of x] (noiser) {$n(x)$};
	
	\node[draw, circle, above right=-.6cm and 1cm of noiser, fill=gray] (x_hat) {$\hat{x}$};
	
	\node[draw, rectangle, above right=0.7cm and 1cm of x_hat] (encoder) {$f(\hat{x})$};
	
	\node[draw, circle, above right=0.7cm and 1cm of encoder, fill=yellow] (h) {$h$};
	
	\node[draw, rectangle, below right=0.7cm and 1cm of h] (decoder) {g(h)};
	
	\node[draw, circle, below right=0.7cm and 1cm of decoder, fill=red] (r) {$r$};
	
	% Arrows
	\draw (x) edge (noiser);
	\draw (noiser) edge (x_hat);
	\draw (x_hat) edge (encoder);
	\draw (encoder) edge (h);
	\draw (h) edge (decoder);
	\draw (decoder) edge (r);
	
	% Loss
	\node[below of=h, yshift=-0.7cm] (loss) {eventual perda na reconstrução};
	\draw [->,in=180,out=-90](x) edge (loss);
	\draw [->,in=0,out=-90] (r) edge (loss);
	
	% Labels
	\node[above of=x, yshift=-2cm] {entrada};
	\node[above of=r, yshift=-2cm, xshift=.5cm] {reconstrução};
\end{tikzpicture}
			\label{fig:denoisingAutoencoder}
		\end{figure}

	}
\end{frame}