\begin{frame}
	\frametitle{A base de sinais}
	\only<1>{
		\framesubtitle{Coleta dos dados}
		\begin{itemize}
			\item questionário preliminar.
			\item ciclo silencioso e ruidoso.
			\item coleta da fala, EEG, EEG e fala.
			\item coleção de sentenças específicas.
			\item quantização de 16 bits para voz e EEG.
			\item taxa de amostragem 44100Hz para voz e 800Hz para EEG.
		\end{itemize}
	}
	
	\only<2>{
		\framesubtitle{Questionário preliminar}
		\begin{itemize}
			\item ingeriu ou ingere algum medicamento? Há quanto tempo?
			\item usou alguma droga legal ou ilegal? Há quanto tempo?
			\item ingeriu alguma bebida energética? Há quanto tempo?
			\item é canhota, destra ou ambidestra?
			\item qual seu gênero?
			\item qual sua etnia?
			\item qual seu sexo?
		\end{itemize}
	}
	
	\only<3>{
		\framesubtitle{Protocolo de coleta da fala fonada}
		\begin{enumerate}
			\item \textbf{ciclo silencioso.}
			\begin{enumerate}
				\item em uma tela é exibido ao participante por 5 segundos a sentença que deve ser pronunciada.
				\item um sinal é tocado por 1 segundo
				\item por 5 segundos a sentença deve ser \textbf{pronunciada} uma única vez.
				\item volta para o item 1.1 até que todas as sentenças sejam pronunciadas.
			\end{enumerate}
			
			\item \textbf{ciclo ruidoso}.
			\begin{enumerate}
				\item inicia-se a reprodução do ruído
				\item em uma tela é exibido ao participante por 5 segundos a sentença que deve ser pronunciada.
				\item um sinal é tocado por 1 segundo
				\item por 5 segundos a sentença deve ser \textbf{pronunciada} uma única vez.
				\item volta para o item 2.2 até que todas as sentenças sejam pronunciadas.
				\item a reprodução do ruído é finalizada.
			\end{enumerate}
		\end{enumerate}
	}
	
	\only<4>{
		\framesubtitle{Protocolo de coleta da fala imaginada}
		
		\begin{enumerate}
	    	\item Coleta "pré-protocolar" de cinco segundos.	
			\item \textbf{ciclo silencioso.}
			\begin{enumerate}
				\item em uma tela é exibido ao participante por 5 segundos a sentença que deve ser imaginada.
				\item um sinal é tocado por 1 segundo
				\item Por 5 segundos a sentença deve ser \textbf{imaginada} uma única vez.
				\item volta para o item 2.1 até que todas as sentenças sejam imaginadas.
			\end{enumerate}
			\item \textbf{ciclo ruidoso}.
			\begin{enumerate}
				\item inicia-se a reprodução do ruído
				\item em uma tela é exibido ao participante por 5 segundos a sentença que deve ser imaginada.
				\item um sinal é tocado por 1 segundo
				\item por 5 segundos a sentença deve ser \textbf{imaginada} uma única vez.
				\item volta para o item 3.1 até que todas as sentenças sejam pronunciadas.
				\item a reprodução do ruído é finalizada.
			\end{enumerate}
		\end{enumerate}
	}
	
	\only<5>{
		\framesubtitle{Protocolo de coleta da fala mista}
		\par Deve seguir os mesmos protocolos para o ciclo silencioso e ruidoso anteriormente descritos porém, captando ambos os sinais fonados e imaginados simultaneamente;
	}

	\only<6>{
		\framesubtitle{Protocolo de coleta de dados}
		\begin{textblock*}{\linewidth}(.3cm,.7cm)
			\begin{figure}[t]
				\centering
				\subfloat[0.5\textwidth][Protocolo de coleta silenciosa]{
					\scalebox{0.6}{
						\begin{tikzpicture}[transform shape]
	% Definindo estilos de nós
	
	% Início
	\tikzstyle{startstop} = [circle, rounded corners, minimum width=1cm, minimum height=1cm, text centered, fill=orange, text=black, font=\sffamily\bfseries]
	
	% Processo
	\tikzstyle{process} = [rectangle, rounded corners, minimum width=3cm, minimum height=1cm, text centered, fill=green, font=\sffamily\bfseries, inner sep=0.4cm]
	
	% Decisão
	\tikzstyle{decision} = [diamond, minimum width=3cm, minimum height=1cm, text centered, fill=yellow, text=black, font=\sffamily\bfseries, aspect=3]
	
	% Estilos das setas
	\tikzstyle{arrow} = [thick,->,>=stealth, draw=gray, line width=1.5mm]
	
	% Nós
	\node (start) [startstop] {Início};
	\node (display) [process, below=1cm of start] {Exibe próxima sentença por 5s};
	\node (signal) [process, right=1cm of display] {Sinal por 1s};
	\node (pronounce) [process, below=1cm of signal] {Captação por 5s};
	\node (loop) [decision, left=1cm of pronounce, below=1.5cm of display] {Existem mais sentenças?};
	\node (stop) [startstop, below=1.5cm of loop] {Fim};
	
	
	% Setas com curvas
	\draw [arrow] (start) to (display);
	\draw [arrow] (display) to (signal);
	\draw [arrow] (signal) to (pronounce);
	\draw [arrow, bend left=20] (pronounce) to (loop.east);
	\draw [arrow, bend left=20] (loop.north) to node[midway, sloped, below, font=\sffamily\bfseries, text=white] {Sim} (display);
	\draw [arrow, bend left=20] (loop.south) to node[midway, sloped, below, font=\sffamily\bfseries, text=white] {Não} (stop);
	
	% Agrupando o diagrama em um escopo
	\begin{scope}[on background layer]
		% Retângulo ao redor do diagrama com fundo azul
		\node[rectangle, fill=none, fit={(start) (display) (pronounce) (loop) (stop)}, inner sep=0.5cm] (background) {};
	\end{scope}
\end{tikzpicture}
					}
					\label{fig:coletaSilenciosa}
				}
				\subfloat[0.5\textwidth][Protocolo de coleta ruidosa]{
					\scalebox{0.6}{
						\begin{tikzpicture}[transform shape, scale=0.6]
	% Definindo estilos de nós
	
	% Início
	\tikzstyle{startstop} = [circle, rounded corners, minimum width=1cm, minimum height=1cm, text centered, fill=orange, text=black, font=\sffamily\bfseries]
	
	% Processo
	\tikzstyle{process} = [rectangle, rounded corners, minimum width=3cm, minimum height=1cm, text centered, fill=green, font=\sffamily\bfseries, inner sep=0.4cm]
	
	% Decisão
	\tikzstyle{decision} = [diamond, minimum width=3cm, minimum height=1cm, text centered, fill=yellow, text=black, font=\sffamily\bfseries, aspect=3]
	
	% Estilos das setas
	\tikzstyle{arrow} = [thick,->,>=stealth, draw=white, line width=1.5mm]
	
	% Nós
	\node (start) [startstop] {Início};
	
	\node (noise) [process, left=1cm of start] {Reproduzir ruído};
	
	\node (display) [process, below=1cm of noise] {Exibe próxima sentença por 5s};
	\node (signal) [process, right=1cm of display] {Sinal por 1s};
	\node (pronounce) [process, below=1cm of signal] {Captação por 5s};
	\node (loop) [decision, left=1cm of pronounce, below=1.8cm of display] {Existem mais sentenças?};
	
	\node (stopNoise) [process, below=1.4cm of loop] {Silenciar ruído};
	
	\node (stop) [startstop, right=1.5cm of stopNoise] {Fim};
	
	
	% Setas com curvas
	\draw [arrow] (start) to (noise);
	\draw [arrow] (noise) to (display);
	\draw [arrow] (display) to (signal);
	\draw [arrow] (signal) to (pronounce);
	\draw [arrow, bend left=20] (pronounce) to (loop.east);
	\draw [arrow, bend left=20] (loop.north) to node[midway, sloped, below, font=\sffamily\bfseries, text=white] {Sim} (display);
	
	\draw [arrow, bend left=20] (loop.south) to node[midway, sloped, below, font=\sffamily\bfseries, text=white] {Não} (stopNoise);
	
	\draw [arrow] (stopNoise) to (stop);
	
	% Agrupando o diagrama em um escopo
	\begin{scope}[on background layer]
		% Retângulo ao redor do diagrama com fundo azul
		\node[rectangle, fill=blue, fit={(start) (display) (pronounce) (loop) (stop)}, inner sep=0.5cm] (background) {};
	\end{scope}
\end{tikzpicture}
					}
					\label{fig:coletaRuidosa}
				}
				\label{fig:coleta}
			\end{figure}
		\end{textblock*}
	}
	\only<7>{
		\framesubtitle{Organização}
	    	\begin{itemize}
				\item fonadas
				\begin{itemize}
					\item com ruído
					\item sem ruído
				\end{itemize}
				\item imaginadas
				\begin{itemize}
					\item com ruído
					\item sem ruído
				\end{itemize}
				\item mista (fonadas e imaginadas simultaneamente)
				\begin{itemize}
					\item com ruído
					\item sem ruído
				\end{itemize}
			\end{itemize}
			
			\par Cada registro deve ser vinculado a um código que deve referenciar os dados da pessoa que o gerou, a data, hora, minuto e segundo em que a fala iniciou e finalizou, a sentença e o tipo de fala (fonada ou imaginada).
	}
\end{frame}