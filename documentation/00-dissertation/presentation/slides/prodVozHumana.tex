\begin{frame}
	\frametitle{Caracterização dos processos de produção da voz humana}
	\only<1>{
		\framesubtitle{Áreas de estudo}
		\begin{itemize}
			\item Fisiológica ou “fonética articulatória”.
			\item Acústica ou “fonética acústica”.
			\item Perceptual.
		\end{itemize}
		\textbf{Foco apenas na acústica}
	}
	\only<2>{
		\framesubtitle{Vozeada versus não-vozeada}
		\begin{itemize}
			\item Vozeada: Pregas vocais.
			\item Não vozeada: Sem pregas vocais.
		\end{itemize}
	}
	\only<3>{
		\framesubtitle{Frequência fundamental da voz}
		\begin{itemize}
			\item Conhecida como $F_0$.
			\item Componente periódico resultante da vibração das pregas vocais.
		\end{itemize}
	}
	\only<4>{
		\framesubtitle{Formantes}
		\begin{itemize}
			\item $F_1 \rightarrow$ amplificação  sonora  na  cavidade  oral  posterior,  posição  da  língua  no  plano  vertical.
			\item $F_2 \rightarrow$ cavidade  oral  anterior,  posição  da  língua  no  plano  horizontal.
			\item $F_3 \rightarrow$ cavidades  à  frente  e  atrás  do  ápice  da  língua.
			\item $F_4 \rightarrow$ formato da laringe e da  faringe.
		\end{itemize}
	}
\end{frame}