\begin{frame}
	\frametitle{Caracterização dos processos de produção da voz humana}
	\only<1>{
		\framesubtitle{Áreas de estudo}
		\begin{itemize}
			\item fisiológica ou “fonética articulatória”.
			\item acústica ou “fonética acústica”.
			\item perceptual.
		\end{itemize}
		\par Neste trabalho, o foco será apenas na questão acústica, pois não serão analisados aspectos da fisiologia relacionada à voz, mas sim os sinais sonoros propriamente ditos.
	}
	\only<2>{
		\framesubtitle{Vozeada versus não-vozeada}
		\begin{itemize}
			\item vozeada: Pregas vocais.
			\item não vozeada: Sem pregas vocais.
		\end{itemize}
	}
	\only<3>{
		\framesubtitle{Frequência fundamental da voz}
		\begin{itemize}
			\item conhecida como $F_0$ representa o tom ou \textit{pitch} da voz \cite{kremer2014eficiencia}.
			\item reflete a excitação pulmonar moldada pelas pregas vocais.
			\item componente periódico resultante da vibração das pregas vocais.
			\item geralmente são apresentadas em Hz \cite{freitas2013avaliaccao}
		\end{itemize}
		
		\vspace{3em}
		
		\par A alteração desta frequência (jitter) e/ou intensidade (shimmer) do \textit{pitch} durante a fala é definida como entonação,  porém, também pode indicar algum distúrbio ou doença relacionada ao trato vocal \cite{WERTZNER2005}.
	}
	\only<4>{
		\framesubtitle{Formantes}
		\par Se referem as modificações feitas em $F_0$ pelas estruturas do sistema fonador \cite{valencca2014analise}:
		\begin{itemize}
			\item $F_1 \rightarrow$ amplificação  sonora  na  cavidade  oral  posterior,  posição  da  língua  no  plano  vertical.
			\item $F_2 \rightarrow$ cavidade  oral  anterior,  posição  da  língua  no  plano  horizontal.
			\item $F_3 \rightarrow$ cavidades  à  frente  e  atrás  do  ápice  da  língua.
			\item $F_4 \rightarrow$ formato da laringe e da  faringe.
		\end{itemize}
	}
\end{frame}