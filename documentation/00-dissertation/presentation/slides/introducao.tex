\begin{frame}
	\frametitle{Introdução}
	
	\only<1>{
		\framesubtitle{Motivações}
		\par A autenticação biométrica tem sido amplamente adotada por fornecer individualização baseada em traços únicos dos indivíduos. No entanto, características físicas que diferem das esperadas pelo sistema podem dificultar ou impedir o acesso de certas pessoas, especialmente aquelas pertencentes a grupos étnicos minorizados ou com deficiências. No caso da autenticação por voz, a exigência de uma palavra-chave ou outro tipo de fonação pode criar barreiras adicionais. Para mitigar esse problema, este trabalho propõe um sistema de autenticação biométrica aprimorado pela fala imaginada.
	}
	\only<2>{		
		\framesubtitle{Objetivos}
		\begin{itemize}
			\item Criar um sistema de autenticação por voz aprimorado pela fala imaginada.
			
			\item Comparar o desempenho das estratégias de extração de características automatizadas com o uso de autoencoders com as que usam \textit{Wavelet-Packet} de tempo discreto e engenharia paraconsistente de características.
			
			\item Criar uma base com dados de autenticação voz, fala imaginada e ambos os sinais simultâneos fornecendo a comunidade cientifica mais uma fonte de dados.
		\end{itemize}
	}
\end{frame}