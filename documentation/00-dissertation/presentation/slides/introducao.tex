\begin{frame}
	\frametitle{Introdução}
	
	\only<1>{
		\framesubtitle{Motivações}
		\par Os \textit{voice spoofing attacks} do tipo \textit{playback speech} constituem o tema deste trabalho, pois:
		\begin{itemize}
			\item Sistemas para verificação de voz são uma alternativa aos métodos atuais de autenticação dos usuários.
			\item Podem servir como mais uma camada adicional de segurança.
			\item Em tempos de emergência sanitária como alternativa ao 	reconhecimento por imagem de irís ou impressões digitais.
			\item Devido ao crescimento da aplicação desta tecnologia mais tentativas de fraude são perpetradas.
		\end{itemize}
	}
	\only<2>{		
		\framesubtitle{Objetivos}
		\begin{itemize}
			\item Encontrar um conjunto de características que demonstrem ser as mais disjuntas possíveis possibilitando a distinção entre locuções genuínas e falsificadas.
			
			\item Essas características devem melhorar o desempenho de classificadores em detectar tentativas de burlar os sistemas de verificação de locutores por voz.
		\end{itemize}
	}
\end{frame}