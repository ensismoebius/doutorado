\begin{frame}
	\frametitle{Estrutura da Estratégia Proposta}
		\only<1>{
			\framesubtitle{Diagrama}
			\begin{textblock*}{\linewidth}(0.3cm,2cm)
				\begin{figure}[h]
					\centering
					\scalebox{0.75}	{
						
		\begin{tikzpicture}
			
			% Estilos das setas
			\tikzstyle{seta} = [thick,->,>=stealth, draw=gray, line width=1.5mm]
			
			% Estilos do processo
			\tikzstyle{processo} = [rectangle, rounded corners, minimum width=3cm, minimum height=1cm, text centered, fill=green, font=\sffamily\bfseries, inner sep=0.4cm]
			
			% Estilos Início / Fim
			\tikzstyle{iniFim} = [circle, rounded corners, minimum width=1cm, minimum height=1cm, text centered, fill=orange, text=black, font=\sffamily\bfseries]
			
			% Sinal de entrada
			\node (entrada)[iniFim]{\shortstack{Sinal de entrada\\ sob análise}}; 
			
			% Pré processamento wavelet
			\node (wavelet)[processo, right=3cm of entrada]{
				\shortstack{pré-processamento \\ e extração de \\ características no \\ domínio \textit{wavelet}}
			}; 
			
			% Pré processamento auto-encoder
			\node (encoder)[processo, right=3cm of entrada, below=.5cm of wavelet]{
				\shortstack{pré-processamento \\ auto-encoders}
			}; 	
			
			
			% Agrupamentos energéticos
			\node (barkMel)[processo, right=1.5cm of wavelet]{
				\shortstack{agrupamento energético \\ nas bandas Bark / Mel}
			}; 	
			
			% Análise paraconsistente
			\node (paraco)[processo, below=1cm of barkMel]{
				\shortstack{análise paraconsistente \\ de características}
			}; 
			
			% Seleção da melhor combinação
			\node (selecao)[processo, below=1cm of paraco]{
				\shortstack{seleção da melhor combinação \\ agrupamento energético + wavelet}
			}; 

			% Classificador
			\node (classificador)[processo, below=1.2cm of encoder]{
				\shortstack{Classificador \\ ResNet-SNN}
			};
			
			% Resultados
			\node (resultado)[iniFim, left=2.8cm of classificador] {
				\textbf{Resultados}
			};
			
			\draw[seta] ([yshift=.5cm]entrada.east)  |- ([yshift=.5cm]wavelet.west) node[near end, above] {Sinal de Voz};
			\draw[seta] ([yshift=-.5cm]entrada.east) |- ([yshift=-.5cm]wavelet.west) node[near end, above] {Sinal de EEG};
			
			\draw[seta] ([xshift=.5cm]entrada.south)  |- ([yshift=.3cm]encoder.west) node[near end, above] {Sinal de Voz};
			\draw[seta] ([xshift=-.5cm]entrada.south) |- ([yshift=-.3cm]encoder.west) node[near end, above] {Sinal de EEG};
			
			
			\draw[seta] ([yshift=.3cm]wavelet.east)  |- ([yshift=.3cm]barkMel.west) node[near end, above] {Voz};
			\draw[seta] ([yshift=-.3cm]wavelet.east) |- ([yshift=-.3cm]barkMel.west) node[near end, above] {EEG};
			
			\draw[seta] ([yshift=.3cm]barkMel.east)  -- ++(1.5cm, 0) |- ([yshift=-.3cm]paraco.east) node[near end, above] {Voz};
			\draw[seta] ([yshift=-.3cm]barkMel.east) -- ++(.8cm, 0) |- ([yshift=.3cm]paraco.east) node[near end, above] {EEG};
			
			
			\draw[seta] ([xshift=.5cm]paraco.south)  -- ([xshift=.5cm]selecao.north) node[midway, right] {Voz};
			\draw[seta] ([xshift=-.5cm]paraco.south) -- ([xshift=-.5cm]selecao.north) node[midway, right] {EEG};
			

			\draw[seta] ([yshift=.3cm]selecao.west)  -- ([yshift=.3cm]classificador.east) node[midway, above] {EEG};
			\draw[seta] ([yshift=-.3cm]selecao.west) -- ([yshift=-.3cm]classificador.east) node[midway, above] {Voz};


			\draw[seta] ([xshift=.5cm]encoder.south)  -- ([xshift=.5cm]classificador.north) node[midway, right] {EEG};
			\draw[seta] ([xshift=-.5cm]encoder.south) -- ([xshift=-.5cm]classificador.north) node[midway, right] {Voz};
			
			
			\draw[seta] (classificador.west) -- (resultado.east);
		\end{tikzpicture}
					}
				\end{figure}
			\end{textblock*}
		}
		\only<2>{
			\framesubtitle{Wavelets inicialmente usadas}
				\begin{itemize}
					\item Haar.
					\item Beylkin com suporte 18.
					\item Vaidyanathan de suporte 24.
					\item Daubechies de	suportes 4 até 76.
					\item Symmlets com suportes 8, 16 e 32.
					\item Coiflets com suportes 6, 12, 18, 24 e 30.
				\end{itemize}
		}
		\only<3>{
			\framesubtitle{Métricas}
				\begin{itemize}
					\item Tabela de confusão.
					\begin{itemize}
						\item EER (Equal Error Rate).
						\item Acurácia e seu respectivo desvio padrão.
					\end{itemize}
					\item pontuação F1
					\item precisão
					\item recall
					\item Erro Médio Quadrático
					\item área sob a Curva ROC
					\item sensitividade
					\item especificidade
				\end{itemize}
		}
%		\only<4>{
%			\framesubtitle{Acurácia e EER}
%				\begin{columns}
%					\column{.5\textwidth}
%					\begin{equation}
%						acuracia = \dfrac{TP + TN}{N} \qquad,
%						\label{eq:calculoDaAcuracia}
%					\end{equation}
%					
%					\column{.5\textwidth}
%					\par São calculadas tabelas de confusão por um número suficiente de vezes até que \textbf{\textit{FAR} = \textit{FRR} = \textit{EER}}, a cada ciclo os vetores de características são comutados de forma aleatória.\newline
%
%					\begin{equation}
%						FAR=\dfrac{FP}{TN+FP} \qquad,
%						\label{eq:FAR}
%					\end{equation}
%					
%					\begin{equation}
%						FRR=\dfrac{FN}{TP+FN} \qquad,
%						\label{eq:FRR}
%					\end{equation}
%				\end{columns}
%		}
\end{frame}